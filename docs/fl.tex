\documentclass{sig-alternate}
\usepackage{url}

\title{FriendlyLocation: User Location Estimation Based on the Social Graph}

\numberofauthors{1} 
\author{
    \alignauthor Jeffrey McGee, James Caverlee
    \affaddr{Department of Computer Science and Engineering, Texas A\&M
    University} \\
    \affaddr{ College Station, TX 77845 USA} \\
    \email{jeffamcgee@tamu.edu, caverlee@cse.tamu.edu}
} 
\begin{document}

\maketitle
\begin{abstract}
We investigate the relationship between 
We use what we have observed about the edges of the social graph to improve a location estimation system.

goal: estimate location of users on twitter
special sauce: people live neare friends, some rels are more important than others
\end{abstract}




\section{INTRODUCTION}
\section{RELATED WORK}
\section{DATA COLLECTION}

Datasets
Houston:
snowball sample starting from a handful of users done in November 2010
visited @mentioned users and rfriends
Took everyone who geocoded to Houston, plus users with two out of three:
    high connectedness using score,
    central standard time,
    unknown location field
Ignored accounts that were protected
160k users - 80k definite, 80k probable


Geocrawl:
Listened to geo API for 10 days in April 2011.
Took all users who posted at least 2 tweets.
Found the median location of the tweets for each user
calculated the median distance from a user's tweets to their median location
removed users with a median distance greater than 50 miles - mostly jobs postings
removed users outside of contintental us bounding box
estimated 130k users

for each of these users, grab their friends, followers, and 100 tweets
Ignore users with no connections
lookup profiles for up to 100 rfriends, friends, followers, and mentioned
For 10 of the profiles we looked up that have identifable place, grab friends, followers, and 100 tweets.

\section{INVESTIGATION}

What type of friendship is closest?

What is the relationship between friendship and distance?
mixed social network and news network
some connections are based on 

Are you closer to people you commmunicate with?















analysis of Houston:
most users stay within 8 miles of home
users tweet along roadways
geolocoated tweets show map starting at houston expanding out
looked at triangles posting distance of two sides on x-y axis
strong concentration at 0,0
compared fake triangles to real triangles:
    real triangles have high concentration near origin
    fake triangles spread out
hou_gis: where locations tend to resolve to
Since people often say they live in a city, there are large clusters.

For houston, I picked a friend for each user and went from there.
In houston data, having stars in common hurts the chances that two users are nearby.  It doesn't matter for the geo data. In Houston, we are only looking at up to 100 miles. Does this matter?


normed_edges.png:
conv>ated>at>rfrd>fol>frd>random
why is rfrd so close to fol in the hou data? Because we know they live near eachother?

tpu_hist: tweets per user is zipfian

diff_gnp_gps.png - shows the error in gnp decoding by comparing it to normal tweets

tri_deg_*.png - The number of triangles formed is NOT a function of the number of edges a user has therefore talking about the ratio of people involved in a type of triangle is stupid.

geo_edges_log.png - same as normed_edges.png

geo_tri_rfrd_sp.png - distance is a binomial distribution, possibly the sum of two log-normal distributions.

geo_local_1_*.png when you take out non-locals the cumulative graph starts to look linear.

geo_local_*.png - after removing the non-local, you still see a hump around 10 miles. Is that just error from the geo-coding and city level data?

local_ratio.png - star v. non-star matters a lot for friend case, but not so much in others
\section{EVALUATION}
\section{FUTURE WORK}

\bibliographystyle{abbrv}
\bibliography{fl} 
\end{document}
